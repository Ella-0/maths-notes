\documentclass{article}
\usepackage{graphicx}
\usepackage{amsmath}
\usepackage{amsfonts} 
% or 
\usepackage{amssymb}

\usepackage{mathtools}

\usepackage{tikz}
\usepackage{pgfplots}
\usepgfplotslibrary{fillbetween}
\usetikzlibrary{patterns}

\pgfplotsset{compat = newest}
\usepackage{cancel}
\usepackage[margin=1in]{geometry}

\begin{document}

\title{Maths Lessons}
\maketitle

\section{Quadratics}
Quadratics are second degree polynomials. They usually contain an $x^2$ and can be written
in the form $ax^2 + bx + c$.
\\
The 4 main methods of solving quadratics are as follows:
\begin{itemize}
	\item Factorise
	\item Quadratic Formula
	\item Complete the Square
	\item Graphically
\end{itemize}

\subsection{Quadratic Formula}
The quadratic formula is used to find the solution to arbitrary quadratics in the form $ax^2 + bx + c$. It is as follows:

\begin{equation}
	\frac{-b \pm \sqrt{b^2 - 4ac}}{2a}
\end{equation}

\subsubsection{Deriving the Quadratic fromula}
The quadratic fromula is derived from completing the square. Completing the square is the first step
\begin{gather*}
	ax^2 + bx + c = 0\\
	a \left (x + \frac{b}{2a} \right )^2 - a\left (\frac{b}{2a} \right )^2 + c = 0 \\
	a \left (x + \frac{b}{2a} \right )^2 - \frac{b^2}{4a} + c = 0 \\
	a \left (x + \frac{b}{2a} \right )^2 = \frac{b^2}{4a} - c \\
	\left (x + \frac{b}{2a} \right )^2 = \frac{b^2}{4a^2} - \frac{c}{a} \\
	\left (x + \frac{b}{2a} \right )^2 = \frac{b^2 - 4ac}{4a^2} \\
	x + \frac{b}{2a} = \pm \sqrt{\frac{b^2 - 4ac}{4a^2}} \\
	x + \frac{b}{2a} = \frac{\pm \sqrt{b^2 - 4ac}}{2a} \\
	x = \frac{-b \pm \sqrt{b^2 - 4ac}}{2a} \\
\end{gather*}

\subsubsection{Discriminant}
The discriminant is part of the quadratic formula that can tell us various properties of the parabola.
The discriminant is as follows:
\begin{equation}
	D = b^2 - 4ac
\end{equation}
When the discriminant is positive we have two solutions:
\begin{equation}
	D > 0
\end{equation}
When the discriminant is equal to $0$ we have one solution:
\begin{equation}
	D = 0
\end{equation}
When the discriminant is negative we have no real solutions:
\begin{equation}
	D < 0
\end{equation}
Instead we have complex solutions in the form $ai$ where $i = \sqrt{-1}$.

\section{Trig}

\begin{equation}
	\sec \theta \equiv \frac{1}{\sin \theta}
\end{equation}


\begin{equation}
	\mathrm{cosec} \theta \equiv \frac{1}{\cos \theta}
\end{equation}

\begin{equation}
	\cot \theta \equiv \frac{1}{\tan \theta}
\end{equation}

\begin{equation}
	\sin^2 \theta  + \cos^2 \theta \equiv 1
\end{equation}

\begin{equation}
	\tan^2 \theta + 1 \equiv \sec^2 \theta
\end{equation}

\begin{equation}
	1 + \cot^2 \theta \equiv \mathrm{cosec}^2 x
\end{equation}

\section{Calculus}

Differentiation, finding a function for the gradient.

\begin{gather*}
	f(x)\\
	m = \frac{y - y_1}{x - x_1}\\
	h = x - x_1 \\
	\text{We want to make $h$ as small as possible so we say that: }\\
	h \to 0 \\
	y = f(x + h) \\
	y_1 = f(x) \\
	f'(x) = \lim_{h \to 0} \frac{f(x + h) - f(x)}{h}
\end{gather*}

\begin{gather*}
	f(x) = x^2 \\
	f'(x) = \lim_{h \to 0} \frac{f(x + h) - f(x)}{h} \\
	f'(x) = \lim_{h \to 0} \frac{(x + h)^2 - x^2}{h} \\
	f'(x) = \lim_{h \to 0} \frac{x^2 + 2hx + h^2 - x^2}{h} \\
	f'(x) = \lim_{h \to 0} \frac{2hx + h^2}{h} \\
	f'(x) = \lim_{h \to 0} 2x + h \\
	f'(x) = 2x \\
\end{gather*}
Integration, finding a function for the area under a curve.
This is the opposite of differentiation so we can reverse it.


\section{Matrices}

\begin{gather*}
	\begin{pmatrix}
		a & b \\
		c & d \\
	\end{pmatrix} \cdot \begin{pmatrix}
		e & f \\
		g & h \\
	\end{pmatrix} = \begin{pmatrix}
		a \times e + b \times g & a \times f + b \times h \\
		c \times e + d \times g & c \times f + d \times h \\
	\end{pmatrix}
\end{gather*}

\begin{gather*}
	m = \begin{pmatrix}
		a & b \\
		c & d \\
	\end{pmatrix}\\
	m \cdot m^{-1} = \begin{pmatrix}
		1 & 0 \\
		0 & 1 \\
	\end{pmatrix}\\
	m^{-1} = \begin{pmatrix}
		10 & 1 \\
		1 & 10 \\
	\end{pmatrix}
\end{gather*}


\section{Binomial Expansion}

To exapand a binomial in the form $(a + b)^n$, we can use pascals triangle to work out the coeficients and exapnd from there.
For example:
\begin{gather*}
	(a + b)^3 \\
	\text{We use the 4th layer of the pascal triangle $1, 4, 4, 1$} \\
	a^3 + 4a^2b + 4ab^2 + b^3 \\
\end{gather*}
To write a more general formula for this we need to remind ourselves of factorials:
\begin{equation}
	n! = n \times (n - 1) \times (n - 2) \times \cdots \times 1
\end{equation}
This tells us combinations for ordering things. E.g, if given $4$ books they can be put in $4!$ different
orders. We might want to choose a set of $r$ values from a set of $n$ numbers. The function that gives us the number of
combinations for this is as follows.
\begin{equation}
	\binom{n}{r} = \prescript{n}{}{\mathbf{C}_r} = \frac{n!}{(n - r)!r!}
\end{equation}
Using this we can define an expression for arbitrary binomial expansion as follows:
\begin{equation}
	(a + b)^n = a^n + \binom{n}{1}a^{n - 1}b + \binom{n}{2}a^{n - 2}b^{2} + \cdots + \binom{n}{r}a^{n - r}b^{r} + \cdots + b^n
\end{equation}
We can also define an expression for the term at index $r$ in a binomial expansion:
\begin{equation}
	\binom{n}{r}a^{n - r}b^{r}
\end{equation}

\section{Linear Graphs}
Linear graphs are graphs in the form $y = mx + c$ and produce a straight line.

\subsection{Paralel and Perpendicular}
Two linear graphs are parallel if their gradients are equal.
\begin{equation}
	m_1 = m_2
\end{equation}
Two linear graphs are perpendicular if their gradents are negative recipricles of eachother. We can express this rule as the following

\begin{equation}
	m_1 \times m_2 = -1
\end{equation}
This works because the any number multiplied by it's reciprical is always $1$

\begin{equation}
	a \times a^{-1} = 1
\end{equation}

\subsection{Suplimentry rules}

\subsubsection{Calculating the Gradient}

\begin{equation}
	m = \frac{y - y_1}{x - x_1}
\end{equation}

\subsubsection{Calculating Y Offset}
\begin{gather*}
	c = y - mx\\
	\therefore\\
	c = y - \frac{y - y_1}{x - x_1}x
\end{gather*}
\subsubsection{Expanded Form}
TODO: I have this in my notes somewhere; needs finishing
\begin{equation}
	y = \frac{y - y_1}{x - x_1}x + \left (y - \frac{y - y_1}{x - x_1}x \right )
\end{equation}



\enddocument
