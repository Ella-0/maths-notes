\documentclass{article}
\usepackage{graphicx}
\usepackage{amsmath}
\usepackage{amsfonts} 
% or 
\usepackage{amssymb}

\usepackage{mathtools}
\usepackage{siunitx}

\usepackage{tikz}
\usepackage{pgfplots}
\usepgfplotslibrary{fillbetween}
\usetikzlibrary{patterns}

\pgfplotsset{compat = newest}
\usepackage{cancel}
\usepackage[margin=1in]{geometry}

\begin{document}

\title{Further Maths}
\maketitle

\section{Complex Numbers}

\subsection{Arithmetic Laws}
\begin{gather}
	\text{given that}\, a,b \in \mathbb{R}\, \text{then}\, (a + bi)\in \mathbb{C} \\
	i^2 = -1 \\
	(a+bi)i = ai - b \\
	(a + bi) + (c + di) = (a + c) + (b + d)i \\
	(a + bi)(a - bi) = a^2 + b^2 \\
	(a + bi) + (a - bi) = 2a
\end{gather}

\subsubsection{Fractions}
Rationalise the denominator
\begin{gather}
	\frac{a + bi}{c + di}
	\times \frac{c - di}{c - di}
	= \frac{(a + bi)(c - di)}{c^d + d^2}
\end{gather}

\subsection{Complex Conjigates}
\begin{gather}
	Z = a + b i \\
	\overline{Z} = Z^* = a - b i
\end{gather}

\subsection{Solving Polynomial Equations}
The quadratic formula can be used to solve quadratics with
complex roots.
\begin{gather}
	x = \frac{-b \pm \sqrt{b^2 - 4ac}}{2a} \\
	D = b^2 - 4ac \\
	\text{when}\, D < 0 \\
	x = \frac{-b \pm i\sqrt{-D}}{2a}
\end{gather}

\begin{gather}
	az^2 + bz + c = 0 \\
	z^2 + \frac{b}{a}z + \frac{c}{a} \\
	\text{$\alpha$ and $\beta$ are roots} \\
	(z - \alpha)(z - \beta) = 0\\
	z^2 - \beta z - \alpha z + d \beta = 0 \\
	z^2 - z(\alpha + \beta) + \alpha \beta = 0 \\
	-\frac{b}{a} = \alpha + \beta \\
	\frac{c}{a} = \alpha \beta
\end{gather}

\begin{gather}
	z^3 - (\alpha + \beta + \gamma)z^2 + (\alpha \beta + \alpha \gamma + \beta \gamma) z - \alpha \beta \gamma \\
	-\frac{b}{a} = \alpha + \beta + \gamma \\
	\frac{c}{a} = \alpha \beta + \beta \gamma + \alpha \gamma \\
	-\frac{d}{a} = \alpha \beta \gamma
\end{gather}

\begin{gather}
	z^4
	- (\alpha + \beta + \gamma + \delta) z^3
	+ (\alpha \beta + \alpha \gamma + \alpha \delta + \beta \gamma + \beta \delta + \gamma \delta) z^2
	- (\alpha \beta \gamma + \alpha \beta \delta + \alpha \gamma \delta + \beta \gamma \delta)z
	+ \alpha \beta \gamma \delta \\
	-\frac{b}{a} = \alpha + \beta + \gamma + \delta \\
	\frac{c}{a} = \alpha \beta + \alpha \gamma + \alpha \delta + \beta \gamma + \beta \delta + \gamma \delta \\
	-\frac{d}{a} = \alpha \beta \gamma + \alpha \beta \delta + \alpha \gamma \delta + \beta \gamma \delta \\
	\frac{e}{a} = \alpha \beta \gamma \delta
\end{gather}

\subsection{Modulas and argument form}
\begin{gather}
	\text{let}\, z = a + bi \\
	\text{modulus of}\, z \\
	r = |z| = \sqrt{a^2 + b^2} \\
	-\pi < \theta \le \pi \\
	\tan \theta = \frac{a}{b} \\
	\theta = \arctan \frac{a}{b}\\
	z = r \angle \theta = r e^{i \cos \theta} =  r(\cos \theta + i \sin \theta) \\
	\text{argument}\, z = \theta \\
	\text{magnitude}\, z = r
\end{gather}

\subsection{Loci an Argand diagram}

\subsubsection{Circle}
You can draw loci in an Argand diagram These are the set of points they obey a given rule.
\\\\
the lucus points that satisfy $|z| = 4$ draw a circle of radius 4 around the origin.
The following is a sketch of $|z - 3 + 5 i|$:
\begin{center}
	\begin{tikzpicture}
		\begin{axis}[
			xmin=-10, xmax=10,
    		ymin=-10, ymax=10,
			axis lines=middle,
			tick style={draw=none},
			xticklabels={,,},
			yticklabels={,,}
		]
			\draw (3,-5) circle [radius=5]; 
		\end{axis}
	\end{tikzpicture}
\end{center}

\subsubsection{Half Line}
Half line is a locus of points satisfying $\arg(z - z_1) = \theta$.
\\\\
The following is a sketch of $\arg(z + 2 + 3i)$
\begin{center}
	\begin{tikzpicture}
		\begin{axis}[
			xmin=-10, xmax=10,
    		ymin=-10, ymax=10,
			axis lines=middle,
			tick style={draw=none},
			xticklabels={,,},
			yticklabels={,,}
		]
			\draw (-2,-3) -- (14,7.16);
		\end{axis}
	\end{tikzpicture}
\end{center}


\subsubsection{Perpendicular bisector}
All the points satisfying $|z -2| = |z + 3i|$

\begin{center}
	\begin{tikzpicture}
		\begin{axis}[
			xmin=-10, xmax=10,
    		ymin=-10, ymax=10,
			axis lines=middle,
			tick style={draw=none},
			xticklabels={,,},
			yticklabels={,,}
		]
			\draw (0,-3) -- (2,0);
			\draw (-4,0.5) -- (4,-3.5);
		\end{axis}
	\end{tikzpicture}
\end{center}


\subsubsection{Cartesian Equations for Complex Locus}

\begin{gather}
	|z - z_1| = r \\
	|x + iy +  a + ib| = r \\
	|(x + a) +i(y + b)| = r \\
	\sqrt{(x + a)^2 + (y + b)^2} = r \\
	(x + a)^2 + (y + b)^2 = r^2 \\
	\text{radius} = r \\
	\text{centre} = (-a, -b) \\
	\\
	\arg(z + z_1) = \theta \\
	\arg(x + iy + a + ib) = \theta \\
	\arg((x + a) + i(y + b)) = \theta \\
	\arctan{\frac{y + b}{x + a}} = \theta \\
	\frac{y + b}{x + a} = \tan \theta \\
	y + b = (x + a) \tan \theta \\
	y = (x + a) \tan \theta - b \\
	x = (y + b) \cot \theta - a \\
	\\
	|z + z_1| = |z + z_2| \\
	|x + iy + a + ib| = |x + iy + c + id| \\
	|(x + a) + i(y + b)| = |(x + c) + i (y + d)| \\
	\sqrt{(x + a)^2 + (y + b)^2} = \sqrt{(x + c)^2 + (y + d)^2} \\
	(x + a)^2 + (y + b)^2 = (x + c)^2 + (y + d)^2 \\
	(x^2 + 2ax + a^2) + (y^2 + 2by + b^2) = (x^2 + 2cx + c^2) + (y^2 + 2dy + d^2) \\
	(2ax + a^2) + (2by + b^2) = (2cx + c^2) + (2dy + d^2) \\
	2ax - 2cx + a^2 - c^2 = 2dy - 2by + d^2 - b^2 \\
	2x(a - c) + a^2 - c^2 = 2y(d - b) + d^2 - b^2 \\
	2(d - b)y = 2(a - c)x + a^2 + b^2 - c^2 - d^2 \\
	y = \frac{2(a - c)x + a^2 + b^2 - c^2 - d^2}{2(d - b)} \\
	y = \frac{a - c}{d - b} x + \frac{a^2 + d^2 - b^2 - c^2}{2(d - b)}
\end{gather}


\section{Work, Energy and Power}
\subsection{Work Done}
$W\,\si{Nm}\,\left (\si{j} \right )$ is Work done, $F\,\si{N}$ is force, $x\,\si{m}$ is distance and $\theta$ is angle
\subsubsection{Constant Force}
\begin{equation}
	W = Fx
\end{equation}

\subsubsection{Non Constant Force}

\begin{equation}
	W = \int^{x}_{0} F \,dx
\end{equation}

\subsubsection{Constant Force in Direction}

\begin{equation}
	W = F x \cos \theta
\end{equation}

\subsection{Power}
$P\,\si{W}$ is power, $F\,\si{N}$ is force, $v\,\si{ms^{-1}}$ is velocity and $x\,\si{m}$ is distance

\begin{gather}
	P = \frac{Fx}{t} \\
	P = Fv
\end{gather}

\subsection{Energy}

\subsubsection{Potential Energy}
Energy that can potentially be used for
bodies motion

\begin{gather}
	E_p = mgh
\end{gather}

\subsubsection{Kinetic Energy}
Energy for bodies motion

\begin{equation}
	E_k = \frac{1}{2}mv^2
\end{equation}

\subsubsection{Conservation of Energy}
The sum of energy in a system is constant provided energy is not transfered out of the system
or into the system

$-\Delta E_p$ is loss of Gravitation Potential Energy, $\Delta E_k$ is gain in Kinetic Energy
and $W$ is Work Done against Air Resistance
\begin{equation}
	-\Delta E_p = \Delta E_k + W
\end{equation}


\section{Hooke's Law}
Natural length $l$, Elastisity $\lambda$, Extension $x$

\begin{gather}
	T = kx \\
	k = \frac{\lambda}{l} \\
	T = \frac{\lambda x}{l}
\end{gather}

\begin{gather}
	E_{\text{GPE}} = E_{\text{k}} + E_{\text{EPE}} \\
	E_{\text{EPE}} = \int_0^x T\, dx = \int_0^x \frac{\lambda x}{l} dx \\
	= \frac{\lambda}{2l}x^2
\end{gather}

\section{Maclurin Series}

\begin{gather}
	\mathrm{f}(x) = a_0 + a_1x + a_2 x^2 + a_3 x^3 + \cdots + a_r x^r \\
	\mathrm{f}(0) = a_0 \\
	\mathrm{f'}(x) = a_1 + 2a_2 x + 3a_3 x^2 + \cdots + r(a_r) x^{r-1} \\
	\mathrm{f''}(x) = 2 \times 1 \times a_2 + 3 \times 2 \times a_3 x + \cdots + x^{r-3} \\
	\cdots \\
	\mathrm{f^r}(x) = r!a_r \\
	\mathrm{f''}(0) = 2!a_2 \implies a_2 = \frac{\mathrm{f''}(0)}{2!} \\
	\mathrm{f''}(0) = 3!a_2 \implies a_3 = \frac{\mathrm{f'''}(0)}{3!} \\
	\mathrm{f''}(0) = r!a_2 \implies a_r = \frac{\mathrm{f^r}(0)}{r!} \\
	\mathrm{f}(x) = \mathrm{f}(0) + \mathrm(f')(0)x + \frac{\mathrm{f''}(0)}{2!}x^2
	+ \frac{\mathrm{f'''}(x)}{3!}x^3 + \cdots + \frac{\mathrm{f^r}(0)}{r!}x^r \\
\end{gather}

\begin{gather}
	\mathrm{f}(x) = \sin x \\
	\mathrm{f}(0) = 0 \\
	\mathrm{f'}(x) = \cos x \\
	\mathrm{f'}(0) = 1 \\
	\mathrm{f''}(x) = - \sin x \\
	\mathrm{f''}(0) = 0 \\
	\mathrm{f'''}(x) = - \cos x \\
	\mathrm{f'''}(0) = - 1
\end{gather}

\begin{gather}
	\mathrm{f}(x) = e^{-\frac{x}{2}} \\
	\mathrm{f}(0) = 1 \\
	\mathrm{f'}(x) = -\frac{1}{2} e^{-\frac{x}{2}} \\
	\mathrm{f'}(0) = -\frac{1}{2} \\
	\mathrm{f''}(x) = \frac{1}{4} e^{-\frac{x}{2}} \\
	\mathrm{f''}(0) = \frac{1}{}4 \\
	\mathrm{f'''}(x) = -\frac{1}{8} e^{-\frac{x}{2}} \\
	\mathrm{f'''}(0) = -\frac{1}{8} \\
	e^{-\frac{x}{2}} \approx 1 - \frac{1}{2}x + \frac{\cfrac{1}{4}}{2!}x^2 \cdots \frac{\left (-\frac{x}{2} \right )^r}{r !}
\end{gather}

\begin{gather}
	\mathrm{f}(x) = \ln (1 - x) \\
	\ln ( 1 - 0 ) = 0 \\
	\mathrm{f'}(x) = -\frac{1}{1-x}   \\
	\mathrm{f'}(0) = - 1
	\mathrm{f''}(x) = -\frac{1}{(1-x)^2} \\
	\mathrm{f''}(0) = - 1
	\mathrm{f'''}(x) = -\frac{2}{(1-x)^3} \\
	\mathrm{f'''}(0) = - 2 \\
	\ln ( 1 - x ) \approx -x - \frac{x^2}{2!} - \frac{2x^3}{3!}
\end{gather}

\begin{gather}
	\mathrm{f}(x) = e^{i\theta} \\
	\mathrm{f}(0) = 1 \\
	\mathrm{f'}(x) =  i e^{i\theta} \\
	\mathrm{f'}(0) =  i \\
	\mathrm{f''}(0) =  -e^{i\theta} \\
	\mathrm{f''}(0) =  -1 \\
	\mathrm{f'''}(0) =  -ie^{i\theta} \\
	\mathrm{f'''}(0) =  -i \\
\end{gather}

3 units of dimention, length, mass and time

\section{Exponential form of complex numbers}

\begin{equation}
	e^{\mathbf{i}\theta} = (\cos \theta + \mathbf{i} \sin \theta)
\end{equation}

\end{document}

