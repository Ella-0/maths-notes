\documentclass[a4paper]{article}
\usepackage{graphicx}
\usepackage{amsmath}
\usepackage{amsfonts} 
% or 
\usepackage{amssymb}
\usepackage[margin=1in]{geometry}

\begin{document}

\title{A-Level Hegarty Prep}
\maketitle

\section{Surds}

\subsection{Simplifying Surds}

\subsubsection{What is a Surd}
A surd is an irrational number that can be represented as a root and a rational coefficient.
they are represented in the form:
\begin{equation}
	a\sqrt[b]{c}
\end{equation}
For example:
\begin{equation}
	2\sqrt[3]{5}
\end{equation}

\subsubsection{Rules of Surds}
Remember the following:
\begin{equation}
	\sqrt{a} \times \sqrt{b} = \sqrt{a \times b}
\end{equation}
For Example:
\begin{equation}
	\sqrt{9} \times \sqrt{4} = \sqrt{9 \times x} = \sqrt{36}
\end{equation}
We know this works because:
\begin{equation}
	3 \times 4 = 6
\end{equation}
This also means that the following is true since division is just multiplication by the reciprocal:
\begin{equation}
	\frac{\sqrt{a}}{\sqrt{b}} = \sqrt{\frac{a}{b}}
\end{equation}
However, be careful about the following:
\begin{equation}
	\sqrt{a} + \sqrt{b} \ne \sqrt{a + b}
\end{equation}
For example:
\begin{equation}
	\sqrt{9} + \sqrt{16} \ne \sqrt{25}
\end{equation}
We know this because:
\begin{equation}
	3 + 4 \ne 5
\end{equation}
And because subtraction is just addition of a negated number the following is true:
\begin{equation}
	\sqrt{a} - \sqrt{b} \ne \sqrt{a - b}
\end{equation}

\pagebreak 

\subsubsection{How to simplify Surds}
The first step when simplifying a surd is to find the highest factor of the number that is also square.

Given the following surd:
\begin{equation}
	\sqrt{20}
\end{equation}

The highest square factor is 4. This means that we can rewrite the surd as the following:
\begin{equation}
	\sqrt{4 \times 5}
\end{equation}
We can then split up the surd into two different surds:

\begin{equation}
	\sqrt{4} \times \sqrt{5}
\end{equation}
Then since 4 is square we can evaluate the square root giving us the following:
\begin{equation}
	2 \sqrt{5}
\end{equation}

\subsection{How to add Surds}
The first step when adding two surds is to simplify the two surds so that they have the same root part:

Given the following sum:
\begin{equation}
	\sqrt{80} + \sqrt{20}
\end{equation}

We can construct the following equivalent expression:
\begin{equation}
	\sqrt{16 \times 5} + \sqrt{4 \times 5} = 4\sqrt{5} + 2\sqrt{5}
\end{equation}

Then we can simply add together the coefficients:
\begin{equation}
	4\sqrt{5} + 2\sqrt{5} = 6\sqrt{5}
\end{equation}

\pagebreak 

\section{Powers and Indices}

\subsection{Laws of Indices}

\subsubsection{Multiplying Indices}

When multiplying indices we can just add together the exponents.
\begin{equation}
	x^{p + q} \equiv x^{p} \times x^{q}
\end{equation}
\\
For example:
\begin{align*}
	x^3 \times x^4 &= \overbrace{
		\underbrace{x \times x \times x}_{x^3} \times
		\underbrace{x \times x \times x \times x}_{x^4}}^{x^7}\\
	&\therefore\\
	x^3 \times x^4 &= x^7
\end{align*}

\subsection{Powers of Indices}
When we have an index to the power of another number we can multiply the two exponents together.

\begin{equation}
	x^{pq} \equiv (x^{p})^{q} \equiv (x^{q})^{p} 
\end{equation}
\\
For example:
\begin{equation}
	(x^4)^3 = x^{4 \times 3} = x^{12}
\end{equation}


\subsubsection{Division}
When dividing indices we can simply subtract the two exponents as they cancel each other out.

\begin{equation}
	x^{p - q} \equiv \frac{x^{p}}{x^{q}}
\end{equation}
\\
For example:
\begin{equation}
	\frac{x^4}{x^3} = \frac
		{\overbrace{x \times x \times x}^\text{cancelled} \times x}
		{\underbrace{x \times x \times x}_\text{cancelled}}
		= \frac{x}{1} = x
\end{equation}

\subsubsection{Rational Exponents}
When the exponent is a rational number, i.e. it can be represented as a fraction,
the index can be represented as a combination of an integer root and power. Where
the numerator becomes the exponent and the denominator becomes the index number of the root.

\begin{equation}
	\label{simple_equation}
	\sqrt[q]{x}^{p} \equiv x^{\frac{p}{q}}
\end{equation}


\begin{equation}
	\label{simple_equation}
	\sqrt[p]{x} \equiv x^{\frac{1}{p}}
\end{equation}

\subsubsection{Zero}
Anything to the power of 0 is always 1 including 0.

\begin{equation}
	\label{simple_equation}
	x^0 = 1
\end{equation}

\subsubsection{Reciprocal}
When the exponent is $-1$, the expression is equivalent to the reciprocal of $x$

\begin{equation}
	\label{simple_equation}
	x^{-1} \equiv \frac{1}{x}
\end{equation}

\subsubsection{Negative Exponents}
When the exponent is negative, the expression is equivalent to the reciprocal of x to the equivalent positive power.

\begin{equation}
	\label{simple_equation}
	x^{-p} \equiv \frac{1}{x^p}
\end{equation}

\subsection{Exponential Equations}

\subsubsection{Solving simple Exponential equations}
when given an equation in the form $a^x = b$ and we
are tasked with working out the value of x we first
need to write $b$ in the form $a^y$. which gives us:
\begin{equation}
	a^x = a^y
\end{equation}
Because of this, $x$ and $y$ must be the same so the
following is true:
\begin{equation}
	x = y
\end{equation}
\\
For example:
\begin{align*}
	5^x &= 125\\
	125 &= 5^3\\
	&\therefore\\
	5^x &= 5^3\\
	&\therefore\\
	x &= 3
\end{align*}

\section{Graphs}

\subsection{Gradient of a Line}
To calculate the gradient of a line we need to choose two points on
a line and subtract the one with the largest $x$ from the one with the largest $x$
and then dividing the $y$ value of the resulting row vector by the $x$ value.
\\\\
Where $a$ is the coordinate with the largest $x$ and $b$ is the coordinate with the smallest $x$ the following is true:
\begin{align*}
	d &= a - b\\
	m &= \frac{d_y}{d_x}
\end{align*}
\\
Given that $(10, 3)$ and $(4, 1)$ are points on a line, we can construct the following example:
\begin{align*}
	d &= (10, 3) - (4, 1) = (6, 2)\\
	m &= \frac{2}{6} = \frac{1}{3}
\end{align*}

\subsection{Equation of a Line}
To calculate the equation of a line the we need to first calculate the gradient using the previous method
then construct an equation in the form $y = mx + c$, substitute in the $x$ and $y$ values of a coordinate on the
line and solve to get $c$
\\\\
Continuing from the previous example:
\begin{align*}
	y &= \frac{x}{3} + c\\
	1 &= \frac{4}{3} + c\\
	c &= 1 - \frac{4}{3} = -\frac{1}{3}\\
	&\therefore\\
	y &= \frac{x}{3} - \frac{1}{3} = \frac{x - 1}{3}
\end{align*}
We can then check our answer against the other coordinate we used to calculate the gradient.

\begin{align*}
	y &= \frac{x - 1}{3}\\
	3 &= \frac{10 - 1}{3}\\
	3 &= \frac{9}{3}\\
	3 &= 3
\end{align*}

\subsection{Mid Point of a Line}
To calculate the mid point of a line we take the sum of the two end points and divide by 2.
\begin{equation}
	p = \frac{a + b}{2}
\end{equation}
\\
For example:
\begin{align*}
	p &= \frac{(10, 3) + (4, 1)}{2}\\
	p &= \frac{(14. 4)}{2}\\
	p &= (7, 2)
\end{align*}

\pagebreak

\section{Distance}
The distance between two points is calculated using Pythagoras $a^2 + b^2 = c^2$.\\
Given the points $A$ and $B$ that construct the line segment $AB$. The length of $AB$ can be expressed as follows:

\begin{equation}
	|AB| = \sqrt{(A_x - B_x)^2 + (A_y - B_y)^2}
\end{equation}
This can be adapted to $n$ dimensional space using the following:

\begin{equation}
	|AB| = \sqrt{\sum_{m = 1}^{n} (A_m - B_m)^2}
\end{equation}
For Example,
Where $A = (3, -6)$ and $B = (7, -2)$ The length of $AB$ can be calculated using the following:

\begin{align*}
	&\text{We can start by calculating the parts for Pythagoras ($a$ and $b$)}\\
	a &= A_x - B_x = 3 - 7 = -4\\
	b &= A_y - B_y = -6 + 2 = -4 \\
	&\text{We can then substitute these values into Pythagoras to get a final length}\\
	|AB| &= \sqrt{a^2 + b^2} = \sqrt{(-4)^2 + (-4)^2} = \sqrt{16 + 16} = \sqrt{32}\\
	&\text{We can then simplify the surd}\\
	|AB| &= \sqrt{32} = 64\sqrt{2}
\end{align*}


\section{Parallel Lines}
Lines are parallel if they have the same gradient. For example $y = 2x + 3$ and $y = 2x - 2$ are parallel because they
have the same gradient of $2$

\subsection{Calculating Equations of Parallel lines}
Given an equation of a line in the form  $y = mx + c_1$, and a point of a another line $p$ that is parallel.
We can write the equation of the second line in the form $y = mx + c_2$.

\begin{align*}
	&\text{We start by substituting $p$ and rearranging to work out $c_2$}\\
	p_y &= mp_x + c_2 \\
	c_2 &= p_y - mp_x \\
	&\text{We can then substitute our value for $c_2$ into the final equation}\\
	y &= mx + (p_y - mp_x)
\end{align*}
For example, given the initial equation of $y = 2x + 3$ and knowing that the second equation passes through $(3, 4)$,
we can calculate the second equation by the following:

\begin{align*}
	&\text{We start by substituting $p$ and rearranging to work out $c_2$}\\
	p_y &= mp_x + c_2 \\
	4 & = 2 \times 3 + c_2 \\
	c_2 &= 4 - 2 \times 3 = -2\\
	&\text{We can then substitute our value for $c_2$ into the final equation}\\
	y &= mx + c_2\\
	y &= 2x -2
\end{align*}
\section{Perpendicular Lines}
Perpendicular lines are like parallel lines except the gradient of one line is the negative reciprocal of the other.

\begin{equation}
	m_2 = -\frac{1}{m_1}
\end{equation}

\subsection{Calculating Equations of Perpendicular lines}
Like with parallel lines, when given an equation of one line, and a point the other line passes through, we can calculate
the equation of the second line.
\begin{align*}
	&\text{We start by calculating the gradient of the second line}\\
	m_2 &= -\frac{1}{m_1}\\
	&\text{Then we substitute the new gradient}\\
	y &= -\frac{x}{m_1} + c_2\\
	&\text{The we substituting $p$ and rearrange to work out $c_2$}\\
	p_y &=  -\frac{p_x}{m_1}+ c_2 \\
	c_2 &= p_y + \frac{p_x}{m_1}\\
	&\text{We can then substitute our value for $c_2$ into the final equation}\\
	y &= -\frac{x}{m_1} + p_y + \frac{p_x}{m_1}\\
	&\text{We can then simplify this to}\\
	y &= -\frac{x + p_x}{m1} + p_y
\end{align*}
For example, given the initial equation of $y = 2x + 3$ and knowing that the second equation passes through $(3, 4)$,
we can calculate the second equation by the following:

\begin{align*}
	&\text{We start by calculating the gradient of the second line}\\
	m_2 &= -\frac{1}{m_1} = -\frac{1}{2}\\
	&\text{Then we substitute the new gradient}\\
	y &= -\frac{x}{2} + c_2\\
	&\text{The we substituting $p$ and rearrange to work out $c_2$}\\
	y &= m_2x + c_2\\
	4 &=  -\frac{3}{2} + c_2 \\
	c_2 &= 4 + \frac{3}{2}\\
	&\text{We can then substitute our value for $c_2$ into the final equation and simplify}\\
	y &= -\frac{x}{2} + 4 + \frac{3}{2}\\
	y &= -\frac{x - 3}{2} + 4\\
	y &= -\frac{x - 11}{2}
\end{align*}
\end{document}
