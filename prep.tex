\documentclass{article}
\usepackage{graphicx}
\usepackage{amsmath}
\usepackage{amsfonts} 
% or 
\usepackage{amssymb}

\begin{document}

\title{A-Level Maths Notes}
\maketitle

\section{Understand and use the laws of indices}

\subsection{Introducting Subsets of Real Numbers}

\subsubsection{Natural Numbers}
Natural Numbers are positive whole numbers. Weather or not Zero is natural
is disputed but the exam boards say it isn't. They are the "Counting Numbers"
 and are represented by the following:

\begin{equation}
    \label{simple_equation}
    \mathbb{N}_0 = \{0, 1, 2, \dots\}
\end{equation}

\begin{equation}
	\label{simple_equation}
    \mathbb{N}_1 = \{1, 2, 3, \dots\}
\end{equation}

\subsubsection{Integers}
Integers are whole numbers. They can be positive or negative  and are represented by the following:

\begin{equation}
    \label{simple_equation}
    \mathbb{Z} = \{\dots, -3, -2, -1, 0, 1, 2, 3, \dots\}
\end{equation}

\subsubsection{Rational Numbers}
Rational Numbers are any real number that can be represented by a fraction. They are represented by the following:

\begin{equation}
	\label{simple_equation}
	\mathbb{Q} = \{ x | x = \frac{a}{b}, a,b \in \mathbb{Z} \text{ and } b \ne 0 \}
\end{equation}

\subsubsection{Real Numbers}
A real number is any number that is not complex or imaginary. They are represented by the following:

\begin{equation}
	\label{simple_equation}
	\mathbb{R} = \{x | -\infty <  x <\infty\}
\end{equation}

\begin{equation}
	\label{simple_equation}
	\mathbb{N} \subset \mathbb{Z} \subset \mathbb{Q} \subset \mathbb{R}
\end{equation}

\subsection{Indices: The Laws of Indices}

\subsection{Multiplying Indices}
\begin{equation}
	\label{simple_equation}
	x^{p + q} \equiv x^{p} \times x^{q}
\end{equation}

\subsubsection{Powers of Indices}

\begin{equation}
	\label{simple_equation}
	x^{pq} \equiv (x^{p})^{q} \equiv (x^{q})^{p} 
\end{equation}

\subsubsection{Division}

\begin{equation}
	\label{simple_equation}
	x^{p - q} \equiv \frac{x^{p}}{x^{q}}
\end{equation}

\subsubsection{Rational Exponents}

\begin{equation}
	\label{simple_equation}
	\sqrt[q]{x}^{p} \equiv x^{\frac{p}{q}}
\end{equation}


\begin{equation}
	\label{simple_equation}
	\sqrt[p]{x} \equiv x^{\frac{1}{p}}
\end{equation}

\subsubsection{Zero}


\begin{equation}
	\label{simple_equation}
	x^0 = 1
\end{equation}

\subsubsection{Reciprical}

\begin{equation}
	\label{simple_equation}
	x^{-1} = \frac{1}{x}
\end{equation}

\subsubsection{Negative Exponents}

\begin{equation}
	\label{simple_equation}
	x^{-p} = \frac{1}{x^p}
\end{equation}


\section{Use and Manipulate Surds}

\subsection{Surds: Introducing Surds and Simplifying Surds}

\begin{equation}
	\label{simple_equation}
	\sqrt{a} \times \sqrt{a} = a
\end{equation}
\begin{equation}
	\label{simple_equation}
	\sqrt{a} \times \sqrt{b} = \sqrt{ab}
\end{equation}
\begin{equation}
	\label{simple_equation}
	\frac{\sqrt{a}}{\sqrt{b}} = \sqrt{\frac{a}{b}}
\end{equation}

\subsection{Adding and subtracting surds}

\begin{equation}
	\label{bad_add}
	\sqrt{a} + \sqrt{b} \ne \sqrt{a + b}
\end{equation}

\begin{equation}
	\label{simple_surd_add}
	a\sqrt{c} + b\sqrt{c} = (a + b)\sqrt{c}
\end{equation}

\begin{equation}
	\label{surd_add}
	a\sqrt{c \times e} + b\sqrt{c \times d} = a\sqrt{c} \times \sqrt{e} + a\sqrt{c} \times \sqrt{d}
	 = (a\sqrt{e} + b\sqrt{d})\sqrt{c}
\end{equation}

\end{document}
