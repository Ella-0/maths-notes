\documentclass{article}
\usepackage{graphicx}
\usepackage{amsmath}
\usepackage{amsfonts} 
% or 
\usepackage{amssymb}

\begin{document}

\title{A-Level Maths Notes}
\maketitle

\section{Understand and use the laws of indices}

\subsection{Introducting Subsets of Real Numbers}

\subsubsection{Natural Numbers}
Natural Numbers are positive whole numbers. Weather or not Zero is natural
is disputed but the exam boards say it isn't. They are the "Counting Numbers"
 and are represented by the following:

\begin{equation}
    \label{simple_equation}
    \mathbb{N}_0 = \{0, 1, 2, \dots\}
\end{equation}

\begin{equation}
	\label{simple_equation}
    \mathbb{N}_1 = \{1, 2, 3, \dots\}
\end{equation}

\subsubsection{Integers}
Integers are whole numbers. They can be positive or negative  and are represented by the following:

\begin{equation}
    \label{simple_equation}
    \mathbb{Z} = \{\dots, -3, -2, -1, 0, 1, 2, 3, \dots\}
\end{equation}

\subsubsection{Rational Numbers}
Rational Numbers are any real number that can be represented by a fraction. They are represented by the following:

\begin{equation}
	\label{simple_equation}
	\mathbb{Q} = \{ x | x = \frac{a}{b}, a,b \in \mathbb{Z} \text{ and } b \ne 0 \}
\end{equation}

\subsubsection{Real Numbers}
A real number is any number that is not complex or imaginary. They are represented by the following:

\begin{equation}
	\label{simple_equation}
	\mathbb{R} = \{x | -\infty <  x <\infty\}
\end{equation}

\begin{equation}
	\label{simple_equation}
	\mathbb{N} \subset \mathbb{Z} \subset \mathbb{Q} \subset \mathbb{R}
\end{equation}

\subsection{Indices: The Laws of Indices}

\subsection{Multiplying Indices}
\begin{equation}
	\label{simple_equation}
	x^{p + q} \equiv x^{p} \times x^{q}
\end{equation}

\subsubsection{Powers of Indices}

\begin{equation}
	\label{simple_equation}
	x^{pq} \equiv (x^{p})^{q} \equiv (x^{q})^{p} 
\end{equation}

\subsubsection{Division}

\begin{equation}
	\label{simple_equation}
	x^{p - q} \equiv \frac{x^{p}}{x^{q}}
\end{equation}

\subsubsection{Rational Exponents}

\begin{equation}
	\label{simple_equation}
	\sqrt[q]{x}^{p} \equiv x^{\frac{p}{q}}
\end{equation}


\begin{equation}
	\label{simple_equation}
	\sqrt[p]{x} \equiv x^{\frac{1}{p}}
\end{equation}

\subsubsection{Zero}


\begin{equation}
	\label{simple_equation}
	x^0 = 1
\end{equation}

\subsubsection{Reciprical}

\begin{equation}
	\label{simple_equation}
	x^{-1} = \frac{1}{x}
\end{equation}

\subsubsection{Negative Exponents}

\begin{equation}
	\label{simple_equation}
	x^{-p} = \frac{1}{x^p}
\end{equation}


\section{Use and Manipulate Surds}

\subsection{Surds: Introducing Surds and Simplifying Surds}

\begin{equation}
	\label{simple_equation}
	\sqrt{a} \times \sqrt{a} = a
\end{equation}
\begin{equation}
	\label{simple_equation}
	\sqrt{a} \times \sqrt{b} = \sqrt{ab}
\end{equation}
\begin{equation}
	\label{simple_equation}
	\frac{\sqrt{a}}{\sqrt{b}} = \sqrt{\frac{a}{b}}
\end{equation}

\subsection{Adding and subtracting surds}

\begin{equation}
	\label{bad_add}
	\sqrt{a} + \sqrt{b} \ne \sqrt{a + b}
\end{equation}

\begin{equation}
	\label{simple_surd_add}
	a\sqrt{c} + b\sqrt{c} = (a + b)\sqrt{c}
\end{equation}

\begin{equation}
	\label{surd_add}
	a\sqrt{c \times e} + b\sqrt{c \times d} = a\sqrt{c} \times \sqrt{e} + a\sqrt{c} \times \sqrt{d}
	 = (a\sqrt{e} + b\sqrt{d})\sqrt{c}
\end{equation}

\section{Quadratic Functions and Graphs}

\subsection{Factorising Quadratics}

\subsubsection{Difference of Two Squares}
The difference of two squares is a rule that can help us
factorise quardratics in the form $ax^2 + b$ by using the
following identity:

\begin{equation}
	(a - b)(a + b) = a^2 + b^2
\end{equation}
The proof of this is fairly straight forward; we simply expand and simplify $(a - b)(a + b)$ to get $a^2 + b^2$.
\\
We can apply this identity when given a quardatic in the form $ax^2 + c$.

\begin{align*}
	&\text{First we rearange to fit the identity}\\
	&(x\sqrt{a})^2 + \sqrt{c}^2\\
	&\text{Then we use the identity to split the expression into factors}\\
	&(x\sqrt{a} - \sqrt{c})(x\sqrt{a} + \sqrt{c})
\end{align*}
For example, we can factorise the quardatic $25x^2 - 16$:

\begin{align*}
	&\text{First we rearange to fit the identity}\\
	&(x\sqrt{25})^2 - \sqrt{16}^2 = (5x)^2 - 4^2\\
	&\text{Then we use the identity to split the expression into factors}\\
	&(5x - 4)(5x + 4)
\end{align*}

\subsubsection{Factorising Quadratics in the form $x^2 + bx + c$}
In order to factorise a quardatic in the form $x^2 + bx + c$,
we need to find two numbers, $e$ and $f$ where $e + f = b$ and
$e \times f = c$ then sub them into the following expression:
\begin{equation}
	(x + e)(x + f)
\end{equation}
\\
For example, if given the quardatic $x^2 + 5x + 6$ we can factorise as shown by the following:
\begin{align*}
	2 \times 3 &= 6 \\
	2 + 3 &= 5 \\
	&\therefore \\
	x^2 + 5x + 6 &= (x + 2)(x + 3)
\end{align*}

\subsubsection{Factorising Quadratics in the form $ax^2 + bx + c$}
Inorder to factorise a quardatic in the form $ax^2 + bx +c$, we must first factor out $a$.
This gives us the following expression:
\begin{equation}
	a\left(x^2 + \frac{bx}{a} + \frac{c}{a} \right)
\end{equation}
\\
This makes it easy to factorise quardatics where both $b$ and $c$ are divisible by $a$
For Example, we can factorise the expression $2x^2 + 10x + 12$ with the following steps:

\begin{gather*}
	\text{We start by factoring out a}\\
	2\left(x^2 + \frac{10x}{2} + \frac{12}{2} \right) = 2(x^2 + 5x + 6)\\
	\text{We then factorise the new quardatic}\\
	2((x+2)(x+3)) = 2(x+2)(x+3)\\
	\text{Often our answer will be required in the form $(x + a)(x + b)$}\\
	(2x + 4)(x + 3) \text{ or } (x + 2)(2x + 6)
\end{gather*}
\\
However in some casese neither $b$ or $c$ are divisible by $a$.
For Example $16x^2 - 8x - 3$

\begin{gather*}
	16\left(x^2 - \frac{8}{16} - \frac{3}{16}\right)
\end{gather*}

\end{document}
