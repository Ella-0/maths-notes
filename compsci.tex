\documentclass{article}
\usepackage{graphicx}
\usepackage{amsmath}
\usepackage{amsfonts} 
% or 
\usepackage{amssymb}

\usepackage{tikz}
\usepackage{pgfplots}
\usepgfplotslibrary{fillbetween}
\usetikzlibrary{patterns}

\pgfplotsset{compat = newest}
\usepackage{cancel}
\usepackage[margin=1in]{geometry}
\usepackage{siunitx}
\usetikzlibrary{snakes}
\usepackage{pgflibrarysnakes}
\usetikzlibrary{decorations}

\begin{document}

\section{Signed and Unsigned}

There are three main ways of representing signed integers in binary

\begin{itemize}
	\item{
		\textbf{Sign and Magnitude} --- To represent a negative number in this form we simply take
		the positive number and change the most significant bit to a $1$. For eample $2_{10}$ is $0010_2$ so
		$-2_{10}$ is $1010_2$.
	}
	\item{
		\textbf{1's Compliment} --- To represent a negative number in this form we simply take the positive number
		and invert all of the bits. For example $2_{10}$ is $0010_2$ so $-2_{10}$ is $1101_2$.
	}
	\item{
		\textbf{2's Compliment} --- To represent a negative number in this form we simply take the 1's compliment form and add $1$.
		For example $2_{10}$ is $0010_2$, so the 1's compliment form of $-2_{10}$ (invert all the bits) is $1101_2$. This means
		the 2's compliment form is $1110_2$ ($1101_2 + 0001_2$).
	}
\end{itemize}

\begin{center}
  \begin{tabular}{ | c | c | c | c | }
    \hline
    Decimal & Sign + Mag & 1's Compliment & 2's compliment \\ \hline
    $-2_{10}$ & $1010_2$ & $1101_2$ & $1110_2$\\ \hline
    $-1_{10}$ & $1001_2$ & $1110_2$ & $1111_2$\\ \hline
    $0_{10}$ & $1000_2$ or $0000_2$ &  $1111_2$ or $0000_2$ & $0000_2$\\ \hline
    $1_{10}$ & $0001_2$ & $0001_2$ & $0001_2$\\ \hline
    $2_{10}$ & $0010_2$ & $0010_2$ & $0010_2$ \\
    \hline
  \end{tabular}
\end{center}

\subsection{Sign and Magnitude Properties}
The sign and magnitude notation is by far the simplest of the notations.
It retains the property of the least significant bit telling us it's
parity, and it has the property of the largest significant bit telling us
the sign. It is the easiest for us humans to convert to. However this has the side
affect of allowing two ways of encoding $0$ ($+0$, $-0$) and adding them is non trivial.

\begin{center}
  \begin{tabular}{ | c | c | c | c | c | }
    \hline
    Decimal & $\pm$ & $4_{10}$ & $2_{10}$ & $1_{10}$\\ \hline
    $-2_{10}$ & $1$ & $0$ & $1$ & $0$\\ \hline
    $-1_{10}$ & $1$ & $0$ & $0$ & $1$\\ \hline
    $-0_{10}$ & $1$ & $0$ & $0$ & $0$\\ \hline
    $0_{10}$ & $0$ & $0$ & $0$ & $0$ \\ \hline
    $1_{10}$ & $0$ & $0$ & $0$ & $1$\\ \hline
    $2_{10}$ & $0$ & $0$ & $1$ & $0$ \\
    \hline
  \end{tabular}
\end{center}

\subsection{1's Compliment Properties}
Like the sign and magnitude notation, 1's compliment retains the property of
the most significant bit being the sign however it doesn't retain the property of 
the least significant bit denoting the parity. There are still two notations of 
$0$ with 1's compliment though it is much simpler for us to add up.

\begin{center}
  \begin{tabular}{ | c | c | c | c | c | }
    \hline
    Decimal & $-7_{10}$ & $4_{10}$ & $2_{10}$ & $1_{10}$\\ \hline
    $-2_{10}$ & $1$ & $1$ & $0$ & $1$\\ \hline
    $-1_{10}$ & $1$ & $1$ & $1$ & $0$\\ \hline
    $-0_{10}$ & $1$ &  $1$ & $1$ & $1$\\ \hline
    $0_{10}$ & $0$ &  $0$ & $0$ & $0$\\ \hline
    $1_{10}$ & $0$ & $0$ & $0$ & $1$\\ \hline
    $2_{10}$ & $0$ & $0$ & $1$ & $0$ \\
    \hline
  \end{tabular}
\end{center}

\break

\subsubsection{Adding}
When adding a 1's compliment binary number we start by adding normally. Then, at the end,
we add the carry bit from the $-7_{10}$ column to the beggining.

\begin{center}
  \begin{tabular}{c | c | c | c | c }
    Decimal & $-7_{10}$ & $4_{10}$ & $2_{10}$ & $1_{10}$\\ \hline
    $3_{10}$ & $0$ & $0$ & $1$ & $1$\\
    $-1_{10}$ & $1$ & $1$ & $1$ & $0$\\ \hline
    & $0$ &  $0$ & $0$ & $1$\\
	& $1$ & $1$ &\\ \hline
	Add cary bit & & & &  $1$ \\
	$2_{10}$ & $0$ & $0$ & $1$ & $0$ \\ \hline
	& & & $1$ &\\
  \end{tabular}
\end{center}

\subsection{2's Compliment Properties}
With twos compliment we retain both the properties of the least significant bit telling us the parity
and the most significant bit telling us the sign. There are no longer two notations of $0$ and 
we can use a standard, unsigned adder to get the sum of two 2's compliment binary numbers.

\begin{center}
  \begin{tabular}{ | c | c | c | c | c | }
    \hline
    Decimal & $-8_{10}$ & $4_{10}$ & $2_{10}$ & $1_{10}$\\ \hline
    $-2_{10}$ & $1$ & $1$ & $1$ & $0$\\ \hline
    $-1_{10}$ & $1$ & $1$ & $1$ & $1$\\ \hline
    $0_{10}$ & $0$ &  $0$ & $0$ & $0$\\ \hline
    $1_{10}$ & $0$ & $0$ & $0$ & $1$\\ \hline
    $2_{10}$ & $0$ & $0$ & $1$ & $0$ \\
    \hline
  \end{tabular}
\end{center}

\subsubsection{Adding}
When adding a 2's compliment binary number we just add the number normally as shown bellow:

\begin{center}
  \begin{tabular}{c | c | c | c | c }
    Decimal & $-7_{10}$ & $4_{10}$ & $2_{10}$ & $1_{10}$\\ \hline
    $3_{10}$ & $0$ & $0$ & $1$ & $1$\\
    $-1_{10}$ & $1$ & $1$ & $1$ & $1$\\ \hline
    $2_{10}$& $0$ &  $0$ & $1$ & $0$\\
	& $1$ & $1$ & $1$\\
  \end{tabular}
\end{center}
\end{document}
